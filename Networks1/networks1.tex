\documentclass{article}
\usepackage{graphicx} % Required for inserting images
\usepackage{amsmath} % For bmatrix environment
\usepackage{bm}
\usepackage{mathtools} % For matrix* environment
\usepackage{amssymb} % For \mathbb
\usepackage{url}

\title{Networks 1}
\author{Haim Lavi, 038712105}
\date{November 2025}

\begin{document}
\setcounter{MaxMatrixCols}{20}
\maketitle
All the code is in the following git repository 

\url{https://github.com/HaimL76/Networks.git}
\section{}
\[k_i=\sum_{j=1}^{N}A_{ij}\]

\[\langle{k}\rangle=\frac{1}{N}\sum_{i=1}^{N}k_i=\frac{1}{N}\sum_{i=1}^{N}\sum_{j=1}^{N}A_{ij}\]
\section{}
The average degree of the neighbors of $i$ is
\[
k_{i,nn}=\frac{1}{k_i}\sum_{j=1}^{N}A_{ij}k_j=\frac{\sum_{j=1}^{N}A_{ij}k_j}{\sum_{h=1}^{N}A_{ih}}=\frac{\sum_{j=1}^{N}A_{ij}\sum_{r=1}^{N}A_{jr}}{\sum_{h=1}^{N}A_{ih}}=\frac{\sum_{j=1}^{N}\sum_{r=1}^{N}A_{ij}A_{jr}}{\sum_{h=1}^{N}A_{ih}}
\]
and then
\[
\langle{k_{nn}}\rangle=\frac{1}{N}\sum_{i=1}^{N}k_{i,nn}
\]
\section{}
We create an array of nodes, arbitrarily enumerated, and use it to create the matrix.

\[\{\overset{1}{426},\overset{2}{345},\overset{3}{365},\overset{4}{153},\overset{5}{245},\overset{6}{165},\overset{7}{358},\overset{8}{121},\overset{9}{452},\overset{10}{143},\overset{11}{131},\overset{12}{272},\overset{13}{546},\overset{14}{369},\overset{15}{171},\overset{16}{782},\overset{17}{888}\}\]\[
\hspace{-25mm}
\begin{bmatrix}
& \bm{426} & \bm{345} & \bm{365} & \bm{153} & \bm{245} & \bm{165} & \bm{358} & \bm{121} & \bm{452} & \bm{143} & \bm{131} & \bm{272} & \bm{546} & \bm{369} & \bm{171} & \bm{782} & \bm{888}\\
\bm{426} & 0 & 1 & 1 & 0 & 1 & 1 & 0 & 1 & 1 & 0 & 0 & 0 & 0 & 0 & 0 & 1 & 0\\
\bm{345} & 1 & 0 & 1 & 1 & 1 & 0 & 0 & 0 & 0 & 0 & 0 & 0 & 0 & 0 & 0 & 0 & 0\\
\bm{365} & 1 & 1 & 0 & 0 & 0 & 0 & 0 & 0 & 0 & 0 & 0 & 0 & 0 & 0 & 0 & 0 & 0\\
\bm{153} & 0 & 1 & 0 & 0 & 0 & 0 & 0 & 0 & 0 & 0 & 0 & 0 & 0 & 0 & 0 & 0 & 0\\
\bm{245} & 1 & 1 & 0 & 0 & 0 & 0 & 0 & 0 & 0 & 0 & 0 & 0 & 0 & 0 & 0 & 0 & 0\\
\bm{165} & 1 & 0 & 0 & 0 & 0 & 0 & 1 & 0 & 0 & 0 & 0 & 0 & 0 & 1 & 0 & 0 & 0\\
\bm{358} & 0 & 0 & 0 & 0 & 0 & 1 & 0 & 0 & 1 & 0 & 0 & 0 & 1 & 0 & 0 & 0 & 0\\
\bm{121} & 1 & 0 & 0 & 0 & 0 & 0 & 0 & 0 & 0 & 1 & 1 & 0 & 0 & 0 & 0 & 1 & 0\\
\bm{452} & 1 & 0 & 0 & 0 & 0 & 0 & 1 & 0 & 0 & 0 & 0 & 1 & 0 & 0 & 0 & 0 & 0\\
\bm{143} & 0 & 0 & 0 & 0 & 0 & 0 & 0 & 1 & 0 & 0 & 0 & 0 & 0 & 0 & 0 & 0 & 0\\
\bm{131} & 0 & 0 & 0 & 0 & 0 & 0 & 0 & 1 & 0 & 0 & 0 & 0 & 0 & 0 & 0 & 0 & 0\\
\bm{272} & 0 & 0 & 0 & 0 & 0 & 0 & 0 & 0 & 1 & 0 & 0 & 0 & 0 & 0 & 1 & 1 & 0\\
\bm{546} & 0 & 0 & 0 & 0 & 0 & 0 & 1 & 0 & 0 & 0 & 0 & 0 & 0 & 0 & 0 & 0 & 0\\
\bm{369} & 0 & 0 & 0 & 0 & 0 & 1 & 0 & 0 & 0 & 0 & 0 & 0 & 0 & 0 & 0 & 0 & 0\\
\bm{171} & 0 & 0 & 0 & 0 & 0 & 0 & 0 & 0 & 0 & 0 & 0 & 1 & 0 & 0 & 0 & 0 & 0\\
\bm{782} & 1 & 0 & 0 & 0 & 0 & 0 & 0 & 1 & 0 & 0 & 0 & 1 & 0 & 0 & 0 & 0 & 1\\
\bm{888} & 0 & 0 & 0 & 0 & 0 & 0 & 0 & 0 & 0 & 0 & 0 & 0 & 0 & 0 & 0 & 1 & 0
\end{bmatrix}
\]
\section{}
\[\begin{matrix*}[l]
k_{1}=7\\
k_{2}=4\\
k_{3}=2\\
k_{4}=1\\
k_{5}=2\\
k_{6}=3\\
k_{7}=3\\
k_{8}=4\\
k_{9}=3\\
k_{10}=1\\
k_{11}=1\\
k_{12}=3\\
k_{13}=1\\
k_{14}=1\\
k_{15}=1\\
k_{16}=4\\
k_{17}=1\end{matrix*}\]
\[\langle{k}\rangle=2.4705882352941178\]
\section{}
\[\begin{matrix*}[l]
k_{1,nn}=3.142857142857143\\
k_{2,nn}=3.0\\
k_{3,nn}=5.5\\
k_{4,nn}=4.0\\
k_{5,nn}=5.5\\
k_{6,nn}=3.6666666666666665\\
k_{7,nn}=2.3333333333333335\\
k_{8,nn}=3.25\\
k_{9,nn}=4.333333333333333\\
k_{10,nn}=4.0\\
k_{11,nn}=4.0\\
k_{12,nn}=2.6666666666666665\\
k_{13,nn}=3.0\\
k_{14,nn}=3.0\\
k_{15,nn}=3.0\\
k_{16,nn}=3.75\\
k_{17,nn}=4.0\end{matrix*}\]
\[\langle{k_{nn}}\rangle=3.6554621848739495\]
Since $2.4705882352941178<3.6554621848739495$, this may suggest that the average person is less popular than his or her friends.
\section{}
We calculate the distance matrix $L=(l_{ij})$ by calculating the sequence $A,A^2,A^3\dots$, and setting $l_{ij}=\min\{k|[A^k]_{ij}>0\}$, for every $1\leq{i,j}\leq{N}$.

We calculate powers of $A$ until there is no change in $L$, which means that we found all the paths of the connected nodes.
\[
\hspace{-25mm}
\begin{bmatrix}
& \bm{426} & \bm{345} & \bm{365} & \bm{153} & \bm{245} & \bm{165} & \bm{358} & \bm{121} & \bm{452} & \bm{143} & \bm{131} & \bm{272} & \bm{546} & \bm{369} & \bm{171} & \bm{782} & \bm{888}\\
\bm{426} & 0 & 1 & 1 & 2 & 1 & 1 & 2 & 1 & 1 & 2 & 2 & 2 & 3 & 2 & 3 & 1 & 2\\
\bm{345} & 1 & 0 & 1 & 1 & 1 & 2 & 3 & 2 & 2 & 3 & 3 & 3 & 4 & 3 & 4 & 2 & 3\\
\bm{365} & 1 & 1 & 0 & 2 & 2 & 2 & 3 & 2 & 2 & 3 & 3 & 3 & 4 & 3 & 4 & 2 & 3\\
\bm{153} & 2 & 1 & 2 & 0 & 2 & 3 & 4 & 3 & 3 & 4 & 4 & 4 & 5 & 4 & 5 & 3 & 4\\
\bm{245} & 1 & 1 & 2 & 2 & 0 & 2 & 3 & 2 & 2 & 3 & 3 & 3 & 4 & 3 & 4 & 2 & 3\\
\bm{165} & 1 & 2 & 2 & 3 & 2 & 0 & 1 & 2 & 2 & 3 & 3 & 3 & 2 & 1 & 4 & 2 & 3\\
\bm{358} & 2 & 3 & 3 & 4 & 3 & 1 & 0 & 3 & 1 & 4 & 4 & 2 & 1 & 2 & 3 & 3 & 4\\
\bm{121} & 1 & 2 & 2 & 3 & 2 & 2 & 3 & 0 & 2 & 1 & 1 & 2 & 4 & 3 & 3 & 1 & 2\\
\bm{452} & 1 & 2 & 2 & 3 & 2 & 2 & 1 & 2 & 0 & 3 & 3 & 1 & 2 & 3 & 2 & 2 & 3\\
\bm{143} & 2 & 3 & 3 & 4 & 3 & 3 & 4 & 1 & 3 & 0 & 2 & 3 & 5 & 4 & 4 & 2 & 3\\
\bm{131} & 2 & 3 & 3 & 4 & 3 & 3 & 4 & 1 & 3 & 2 & 0 & 3 & 5 & 4 & 4 & 2 & 3\\
\bm{272} & 2 & 3 & 3 & 4 & 3 & 3 & 2 & 2 & 1 & 3 & 3 & 0 & 3 & 4 & 1 & 1 & 2\\
\bm{546} & 3 & 4 & 4 & 5 & 4 & 2 & 1 & 4 & 2 & 5 & 5 & 3 & 0 & 3 & 4 & 4 & 5\\
\bm{369} & 2 & 3 & 3 & 4 & 3 & 1 & 2 & 3 & 3 & 4 & 4 & 4 & 3 & 0 & 5 & 3 & 4\\
\bm{171} & 3 & 4 & 4 & 5 & 4 & 4 & 3 & 3 & 2 & 4 & 4 & 1 & 4 & 5 & 0 & 2 & 3\\
\bm{782} & 1 & 2 & 2 & 3 & 2 & 2 & 3 & 1 & 2 & 2 & 2 & 1 & 4 & 3 & 2 & 0 & 1\\
\bm{888} & 2 & 3 & 3 & 4 & 3 & 3 & 4 & 2 & 3 & 3 & 3 & 2 & 5 & 4 & 3 & 1 & 0
\end{bmatrix}
\]
\[\langle{l}\rangle=2.676470588235294\]

\section{}
Denote a triangle by $\{i,j,h\}$, where $1\leq{i,j,h}\leq{N}$ are three distinct nodes.

We have $N$ choices for $i$, then $N-1$ choices for $j$ and $N-2$ choices for $h$.

But we can choose every three nodes $i,j,h$ in $3!=6$ different ways, hence $T_{max}=\frac{N(N-1)(N-2)}{3!}=\binom{N}{3}$.

To calculate the actual number of triangles, we observe that if $T=\{i,j,h\}$ is a triangle, then $A_{kl}=1$, where $k,l\in\{i,j,h\}$ and $k\neq{l}$. And then

$A_{ii}=0$, because we do not have loops from a node to itself in a network.

$[A^2]_{ii}=\sum_{k=1}^{N}A_{ik}A_{ki}$, which is the number of paths of length $2$ from node $i$ to itself, which equals to the number of neighbors of node $i$.

$[A^3]_{ii}=\sum_{k=1}^{N}A_{ik}[A^2]_{ki}=\sum_{k=1}^{N}A_{ik}(\sum_{l=1}^{N}A_{kl}A_{li})=\sum_{k=1}^{N}\sum_{l=1}^{N}A_{ik}A_{kl}A_{li}$, which is the count of all the paths from node $i$ to itself of length 3, that is, the count of all the triangles that start and end at node $i$. But we count both paths $i\rightarrow{j}\rightarrow{h}$ and $i\rightarrow{h}\rightarrow{j}$, so we need to divide the count by $2$. To count all the triangles, we must add the counts of all $[A^3]_{kk}$, where $1\leq{k}\leq{N}$.

But if $\{i,j,h\}$ are nodes in a triangles, then $[A^3]_{ii}+[A^3]_{jj}+[A^3]_{hh}$ counts the same triangle three times, because the same triangle can start and end at node $i$ or node $j$ or node $h$, so we need to divide the count by $3$, so the formula is \[
T=\frac{\sum_{i=1}^{N}[A^3]_{ii}}{3!}.
\]
Indeed, in our class network we have $3$ triangles (calculated in the code), and from the diagram we see that the triangles are $\{426,345,365\}$, $\{426,345,245\}$ and $\{426,121,782\}$, and we can see that these are the only nodes for which $[A^3]_{ii}>0$, and $\frac{[A^3]_{ii}}{2}$ is the count of the triangles that node $i$ participates in.

We can see that $\frac{T}{T_{max}}=\frac{\sum_{i=1}^{N}[A^3]_{ii}}{\binom{N}{3}}$, which is our case is $\frac{3}{\binom{17}{3!}}=\frac{18}{17\cdot16\cdot15}$, which is less than half percent of $T_{max}$.
\section{}
We saw in class that the probability of a node $i$ not to be in the graph GCC is $\mathbb{P}[i\notin{GCC}]=1-s=e^{-\langle{k}\rangle{s}}$. Denote by $G_1,G_2$ two GCCs in our network, both with $s=\frac{|G_1|}{N}=\frac{|G_2|}{N}$. The probability of node $i$ to be in $G_1$ is $p_1=\mathbb{P}[i\in{G_1}]=s$, and the probability of $i$ not to be in $G_2$ is $p_2=\mathbb{P}[i\notin{G_2}]=1-s$. Since in $G(n,p)$ the existence of any edge is independent of the existence of other edges, then also the probability to be in $G_1$ is independent of the probability not to be in $G_2$, hence $\mathbb{P}[i\in{G_1}\land{i\notin{G_2}}]=p_1p_2=p_1(1-p_1)=s(1-s)=(1-e^{-\langle{k}\rangle{s}})e^{-\langle{k}\rangle{s}}$, but this decays to zero as $\langle{k}\rangle$ grows. We also need to consider the event where node $i$ is in $G_2$ but not in $G_1$, which is a distinct event with the same probability, hence $\mathbb{P}[(i\in{G_1}\land{i}\notin{G_2})\lor(i\in{G_2}\land{i}\notin{G_1})]=2(1-e^{-\langle{k}\rangle{s}})e^{-\langle{k}\rangle{s}}$, which behaves the same.
\section{}
Both plots are generated in the code.

\textbf{For the first plot}, we use the equation $s=1-e^{-\langle{k}\rangle{s}}$. We run an iteration on a range of increasing discrete values of $\langle{k}\rangle$, where $0\leq\langle{k}\rangle\leq{6}$. We denote by $\langle{k}\rangle_{i}$ the values of $\langle{k}\rangle$ on each iteration.

For each $\langle{k}\rangle_i$, we calculate the matching $s_i$ by running an inner iteration on a range of increasing discrete values of $s_i$. We denote by $s_{i_j}$ the values of $s_i$ on each iteration. For each $s_{i_j}$, we denote by $s'_{i_j}$ the result of the calculation $s'_{i_j}=1-e^{-\langle{k}\rangle{s_{i_j}}}$, and we calculate the difference $d_{i_j}=|s'_{i_j}-s_{i_j}|$. For each $\langle{k}\rangle_i$, we take $s_i$ to be the $s_{i_j}$ with the minimal difference $d_{i_j}$, which we consider the optimal $s_i$ for this $\langle{k}\rangle_i$, and we construct the first plot with these $(\langle{k}\rangle_i, s_i)$ pairs.

\textbf{The second plot} is obtained by running on a range of increasing discrete values of $p$, where $p$ starts from zero and stops after it yields $\langle{k}\rangle=6$. For each $p_i$, we construct a random $G(n,p_i)$ graph (actually we create $m$ different graphs for the same $p_i$, as suggested in the hint), and we calculate the GCC size for each graph. I used a simple recursion to get the neighbors of each node, then the neighbors of the neighbors and so on, which results in lists of connected components (list can contain several GCCs for $\langle{k}\rangle_i<1$, but tends to contain only one GCC for $\langle{k}\rangle_i\geq{1}$), and we take the size $s_i$ of the largest (or only) one, and so we construct the second plot with these pairs $(\langle{k}\rangle_i,s_i)$. I preferred to leave $s$ in the second plot as the absolute size of the largest (or only) GCC size, instead of the fraction $\frac{|GCC|}{N}$.
\begin{figure}[p]
    \makebox[\linewidth]{
        \includegraphics[width=1.95\linewidth]{gcc_graph.png}
    }
\end{figure}
\begin{figure}[p]
    \makebox[\linewidth]{
        \includegraphics[width=1.85\linewidth]{random_graph.png}
    }
\end{figure}
\end{document}
