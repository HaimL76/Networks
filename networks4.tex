\documentclass{article}
\usepackage{graphicx} % Required for inserting images
\usepackage{amsmath}
\title{networks 4}
\author{Haim Lavi, 038712105}
\date{February 2026}

\begin{document}

\maketitle

\section{}
\subsection{}
For every $\frac{1}{t}$, $\alpha$ infected individuals become recovered, regardless of their contact with other individuals, hence $I\rightarrow-\alpha{I}$.\\
For every $\frac{1}{t}$, $\beta$ susceptible individuals become infected, hence $S\rightarrow-\beta{S}\frac{I}{N}$.\\
Susceptible individuals cannot become recovered.\\
Thus, $\frac{dS}{dt}=-\beta{S}\frac{I}{N}$.\\
For every $\frac{1}{t}$, $\beta$ susceptible individuals become infected, provided that they contact infected individuals, but the probability to meet an infected individual is $\frac{I}{N}$, hence $S\rightarrow\beta{S}\frac{I}{N}$.\\
Thus, $\frac{dI}{dt}=-\alpha{I}+\beta{S}\frac{I}{N}$.\\
For every $\frac{1}{t}$, $\alpha$ infected individuals become recovered, regardless of their contact with others, hence $R\rightarrow\alpha{I}$.\\
Thus, $\frac{dR}{dt}=\alpha{I}$.\\
The sum of derivatives is $\frac{dS}{dt}+\frac{dI}{dt}+\frac{dR}{dt}=-\beta{S}\frac{I}{N}-\alpha{I}+\beta{S}\frac{I}{N}+\alpha{I}=0$.\\
Normalized equations are
$\frac{ds}{dt}=-\beta{s}j$, $\frac{dj}{dt}=-\alpha{j}+\beta{s}j$ and $\frac{dr}{dt}=\alpha{j}$, which also sum up to zero.
\subsection{}
$\frac{dj}{dt}=-\alpha{j}+\beta{s}j$, like in the SIS model.
Intuitively speaking, the healthy state is stable, if we calculate the percentage of infected individuals $j$s, because like in the SIS model, a $\delta$ of infected will recover before they infect others, only in this model they will become recovered rather than susceptible.\\
But in the SIR model, $s$ is not the complement of $j$ but rather of $j+r$, hence 
$\frac{dj}{dt}=-\alpha{j}+\beta{s}j=-\alpha{j}+\beta{j}(1-j-r)=f(j,r)$, so we need also the equation
$\frac{dr}{dt}=\alpha{j}$, meaning that $\frac{dr}{dt}$ does not depend on $r$, but then $\frac{d\delta{r}}{dt}=\alpha\delta{j}$.\\
$\frac{d\delta{j}}{dt}=\frac{df}{dr}(j_0,\delta{r})+\frac{df}{dj}(\delta{j},r_0)=-\beta{j_0}\delta{r}+\frac{df}{dj}(\delta{j},r_0)=0+\frac{df}{dj}(\delta{j},r_0)=\frac{d\delta{j}}{dt}$, hence the $j$ perturbation in the SIR model is identical to the SIS model.
\subsection{}
This is also intuitive, because in the first stages of the spread from a state of $s=1,j=0,r=0$, until a recovery occurs (meaning until $\alpha$ time units pass), there is no difference between SIS and SIR, because at this point we have $j$ infected and $s=1-j$ susceptible. After $\alpha$ time units $I\rightarrow{R}$ in the SIR, while $I\rightarrow{S}$ in SIS. Hence \[\frac{dj}{dt}=-\alpha{j}+\beta{j}s=\begin{cases}
-\alpha{j}+\beta(1-r-j)j,&\text{SIR}\\
-\alpha{j}+\beta(1-j)j,&\text{SIS}
\end{cases}\]
But $r\geq{0}\implies-\alpha{j}+\beta(1-r-j)j\leq-\alpha{j}+\beta(1-j)j$, so when recovery appears in the system, the growth of infected in time for SIR is less or equal to that of SIS, but while $r=0$, the growth of infected is identical between the two cases.
\section{}
\subsection{}
From what I read, measles behaves as a SIR model disease.
$\frac{dj}{dt}=-\alpha{j}+\beta{j}s$.
The chart is given in time units of a month.
We look at the population of all the individuals who got infected, that is $N=3,964$.
Looking at April-May 2018:\\
$\frac{32-7}{3964}=\frac{25}{3964}=\frac{dj}{dt}(j=\frac{7}{3964})=-\alpha{\frac{7}{3964}}+\beta\frac{7}{3964}\frac{3964-7}{3964}$.\\
Looking at May-June 2018:\\
$\frac{40-32}{3964}=\frac{8}{3964}=\frac{dj}{dt}(j=\frac{32}{3964})=-\alpha{\frac{32}{3964}}+\beta\frac{32}{3964}\frac{3964-32}{3964}$.\\
This results in $\alpha\approx522.144$ and $\beta\approx526.645$, hence $r=\frac{\beta}{\alpha}=1+\epsilon$.
\subsection{}
At $r$ slightly greater than $1$ we expect an immediate rapid growth in infected individuals in the first stages of the spread, instead we have a mild growth, which may suggest an efficient vaccination, or other efficient ways to treat the disease such as isolation.
\end{document}
