\documentclass{article}
\usepackage{graphicx} % Required for inserting images
\usepackage{float} % Allows use of [H] float option
\usepackage{amsfonts} % For \mathbb
\usepackage{amsmath}

\title{Networks 3}
\author{Haim Lavi, 038712105}
\date{December 2025}

\begin{document}

\maketitle

\section{}
Code is in https://github.com/HaimL76/Networks.git
\section{}
All the plots in this section are for the BA model without fitness.
\subsection{}
Graph of the average degree as function of the time / number of nodes.
\begin{figure}[H]
    \makebox[\linewidth]{
\includegraphics[width=1.85\linewidth]{ba_model_k_average_n.png}
    }
\end{figure}
\subsection{}
Graph of the degree ratio as function of the square root of the time / number of nodes.
\begin{figure}[H]
    \makebox[\linewidth]{
\includegraphics[width=1.75\linewidth]{ba_model_k_ratio_n.png}
}
\end{figure}
\subsection{}
Graph of the degree of several nodes as function of the time / number of nodes (in log-log scale).
\begin{figure}[H]
    \makebox[\linewidth]{
\includegraphics[width=1.75\linewidth]{ba_model_k_i_sqrt_t_loglog.png}
    }
\end{figure}
\subsection{}
Graph of the degree distribution (in log-log scale, with the calculated distribution).
\begin{figure}[H]
    \makebox[\linewidth]{
\includegraphics[width=1.75\linewidth]{ba_model_p_k_loglog_with_slope.png}
    }
\end{figure}
Graph of the degree distribution (in log-log scale, with log binning, with the calculated distribution).
\begin{figure}[H]
    \makebox[\linewidth]{
\includegraphics[width=1.75\linewidth]{ba_model_p_k_loglog_binning_with_slope.png}
    }
\end{figure}
Graph of the degree distribution (in log-log scale, with log binning taking medians only, with the calculated distribution).
\begin{figure}[H]
    \makebox[\linewidth]{
\includegraphics[width=1.75\linewidth]{ba_model_p_k_loglog_binning_with_slope_take_medians.png}
    }
\end{figure}
\section{}
We use the same analysis method, $\frac{dk_i}{dt}=\mathbb{E}[k_i(t)]=m\pi_i(t)=m\frac{k_i\eta_i}{\sum_{j=1}^tk_j\eta_j}$, where $\eta_i$ is fixed for every node $i$ (the node that appears at $t_i$). Taking $\eta_i=\alpha$ for all $i$, where $\alpha$ is some constant, reduces to the BA preferential attachment model. Taking $\eta_i>\eta_j$ for every $i<j$ only increases the first-mover advantage in this model, because then $\pi_i>>\pi_j$, so $\eta_i$ is not a valid fitness. So we must take $\eta_j>\eta_i$ for every $i<j$. I did not achieve a general analysis of this, so for simplicity
assume $\sum_{j=1}^tk_j\eta_j=\alpha{t}$ for all $t$, where $\alpha$ is some constant, and fix $\eta_i=\beta{t_i}$, for some constant $\beta$. so $\frac{dk_i}{dt}=m\frac{k_i\eta_i}{\alpha{t}}=m\frac{k_i\beta{t_i}}{\alpha{t}}$, but $\eta_i$ is fixed for every $i$, so we have a differential equation which translates to $\frac{dk_i}{k_i}=\frac{m\beta{t_i}}{\alpha}\frac{dt}{t}$, which translates to $\log(\frac{k_i}{m})=\frac{m\beta{t_i}}{\alpha}\log(\frac{t}{t_i})$, so $\log(\frac{k_{i+1}}{m})=\frac{m\beta{t}_{i+1}}{\alpha}\log(\frac{t}{t_{i+1}})=\frac{m\beta(t_i+1)}{\alpha}\log(\frac{t}{t_{i+1}})=\frac{m\beta{t_i}+m\beta}{\alpha}\log(\frac{t}{t_{i+1}})$, so the log-log slope for $k_{i+1}$ is greater than the slope for $k_i$ in $\frac{m\beta}{\alpha}$. We choose appropriate $\alpha$ and $\beta$, and the increase in slopes, for every $j>i$ ensures that we have a range of nodes $j_1,j_2,\dots>i$ s.t. $k_{j_1},k_{j_2}\geq{k_i}$, but we will always have a tail of late-mover nodes that cannot match their predecessors (because then it means their slope is infinite).
\end{document}
