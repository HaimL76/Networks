\documentclass{article}
\usepackage{graphicx} % Required for inserting images
\usepackage{float} % For the [H] float option
\usepackage{amsmath}
\title{Networks2}
\author{haiml76 }
\date{December 2025}

\begin{document}

\maketitle
\section{}
\section{}
\section{}
For this plot we construct a random ER network ($N=10^4$ and $p=10^{-3}$). We create the vector $(k_1,k_2,\dots,k_{N})$ by calculating $k_i=\sum_{j=1}^{N}A_{ij}$ for every $1\leq{i}\leq{N}$. Then we count the number $N_i$ of nodes whose degree is $k_i$ and calculate $P(k_i)=\frac{N_i}{N}$. We plot this graph in blue.

We calculate the Poisson distribution $P(k)=\frac{e^{-\lambda}\lambda^k}{k!}$, where $\lambda=pN$, for every $0\leq{k}\leq\max\{k_i\}$, and plot this graph in red.

We can see that the two graphs are very similar to each other.
\begin{figure}[H]
    \makebox[\linewidth]{
\includegraphics[width=1.85\linewidth]{degree_distribution.png}
    }
\end{figure}
\section{}
\subsection{}
\[
P(k)=Ck^{-\gamma}
\]
But
\[
C=\frac{\gamma-1}{k_{\min}^{1-\gamma}}
\]
But in our case $k_{\min}=1$, hence $C=\frac{\gamma-1}{1^{1-\gamma}}=\gamma-1$
\[
E(k^n)=\int_1^{\infty}k^n(\gamma-1)k^{-\gamma}dk=\int_1^{\infty}(\gamma-1)k^{n-\gamma}dk=\frac{\gamma-1}{n-\gamma+1}k^{n-\gamma+1}\bigg\rvert_1^{\infty}
\]

If $n-\gamma\geq-1$ then $n-\gamma+1\geq{0}$ and then $k^{n-\gamma+1}\overset{k\rightarrow\infty}{\longrightarrow}\infty$, so $E(k^n)$ diverges for every $\gamma\leq{n+1}$. If $\gamma>n+1$ then $k^{n-\gamma+1}\overset{k\rightarrow\infty}{\longrightarrow{0}}$, so $E(k^n)$ is finite.
\subsection{}
The mean is $\mu=E(k)=E(k^1)$, so it diverges for $\gamma\leq{1+1}=2$ and converges for $\gamma>2$.

The variance is $\sigma^2=E(k^2)$, so it diverges for $\gamma\leq{2+1}=3$ and converges for $\gamma>3$.
\end{document}
