\documentclass{article}
\usepackage{graphicx} % Required for inserting images
\usepackage{float} % For the [H] float option
\usepackage{amsmath}
\usepackage{amssymb}
\usepackage{xfrac}
\title{Networks2}
\author{haiml76 }
\date{December 2025}

\begin{document}

\maketitle
\section{}
For $d=1$, it is obvious that no GCC can be formed, because in that case we have only single edges in the network. But for any $d\geq{2}$, every node can be connected to the GCC.

The equation is
$S=1-e^{-dS}$
\section{}
\subsection{}
If we have $N$ nodes, then $0\leq{d}\leq{N-1}$.

We calculate $\langle{k}\rangle$ by calculating the mean of $d$.

For every node $0\leq{i}\leq{N}$, the probability for $d_i=1$ is that $i$ has only one connected node, that is, node $i$ is connected 
\section{}
For this plot we construct a random ER network ($N=10^4$ and $p=10^{-3}$). We create the vector $(k_1,k_2,\dots,k_{N})$ by calculating $k_i=\sum_{j=1}^{N}A_{ij}$ for every $1\leq{i}\leq{N}$. Then we count the number $N_i$ of nodes whose degree is $k_i$ and calculate $P(k_i)=\frac{N_i}{N}$. We plot this graph in blue.

We calculate the Poisson distribution $P(k)=\frac{e^{-\lambda}\lambda^k}{k!}$, where $\lambda=pN$, for every $0\leq{k}\leq\max\{k_i\}$, and plot this graph in red.

We can see that the two graphs are very similar to each other.
\begin{figure}[H]
    \makebox[\linewidth]{
\includegraphics[width=1.85\linewidth]{degree_distribution.png}
    }
\end{figure}
\section{}
\subsection{}
\[
P(k)=Ck^{-\gamma}
\]

\[
E(k^n)=\int_1^{\infty}k^nCk^{-\gamma}dk=C\int_1^{\infty}k^{n-\gamma}dk=\frac{C}{n-\gamma+1}k^{n-\gamma+1}\bigg\rvert_1^{\infty}
\]

But
\[
C=\frac{\gamma-1}{k_{\min}^{1-\gamma}}
\]
In our case $k_{\min}=1$, hence $C=\frac{\gamma-1}{1^{1-\gamma}}=\gamma-1$, hence 

\[
E(k^n)=\frac{\gamma-1}{n-\gamma+1}k^{n-\gamma+1}\bigg\rvert_1^{\infty}
\]

If $n-\gamma\geq-1$ then $n-\gamma+1\geq{0}$ and then $k^{n-\gamma+1}\overset{k\rightarrow\infty}{\longrightarrow}\infty$, so $E(k^n)$ diverges for every $\gamma\leq{n+1}$. If $\gamma>n+1$ then $k^{n-\gamma+1}\overset{k\rightarrow\infty}{\longrightarrow{0}}$, so $E(k^n)$ is finite.
\subsection{}
The mean is $\mu=E(k)=E(k^1)$, so it diverges for $\gamma\leq{1+1}=2$ and converges for $\gamma>2$.

The variance is $\sigma^2=E(k^2)-E(k)^2$, so it diverges for $\gamma\leq{2+1}=3$ and converges for $\gamma>3$ (because then both $E(k)$ and $E(k^2)$ converge).
\section{}
\subsection{}
Minimum wage $x_{\min}=6,247.67\approx{6,248}$ ILS

Average wage $\overline{x}=13,620$ ILS

\[
\mu=E(x)=\frac{C}{2-\gamma}x^{2-\gamma}\bigg\rvert_{6248}^{\infty}\approx\overline{x}=13620
\]

But

\[
C=\frac{\gamma-1}{x_{\min}^{1-\gamma}}=\frac{\gamma-1}{6248^{1-\gamma}}
\]

Hence

\[
\frac{\gamma-1}{\gamma-2}\frac{6248^{2-\gamma}}{6248^{1-\gamma}}=\frac{\gamma-1}{\gamma-2}\frac{6248^{1-\gamma}6248}{6248^{1-\gamma}}=\frac{\gamma-1}{\gamma-2}6248\approx{13620}
\]

hence

\[
\frac{\gamma-1}{\gamma-2}\approx{2.18}
\]

hence

\[
\gamma-1\approx{2.18(\gamma-2)}=2.18\gamma-4.36
\]

hence

\[
1.18\gamma\approx{3.36}
\]

hence $\gamma\approx{2.847}$
\subsection{}
\[
x_{\max}\approx{N^{\frac{1}{\gamma-1}}}\approx(10^6)^{\frac{1}{2.847-1}}=10^{\frac{6}{1.847}}\approx{10^{3.248}}
\]

But this is an obvious mistake, so I will take the value of $\gamma$ to be around $2$, hence $x_{\max}\approx({10}^6)^{\frac{1}{2-1}}=10^6$.
\subsection{}
The wage distribution is close enough to continuous distribution, hence we need to find $x_{\sfrac{1}{2}}$ s.t. $P(x<x_{\sfrac{1}{2}})=\frac{1}{2}$, then $x_{\sfrac{1}{2}}$ is the median value.

\[
P(k<x_{\sfrac{1}{2}})=\int_{x_{\min}}^{\sfrac{1}{2}}Ck^{-\gamma}dk=\frac{C}{1-\gamma}k^{1-\gamma}\bigg\rvert_{x_{\min}}^{x_{\sfrac{1}{2}}}=\frac{1}{2}
\]

But in our case, $\gamma\approx2$, hence $C=\frac{\gamma-1}{x_{\min}^{1-\gamma}}=\frac{1}{6248^{-1}}=6248$, hence

\[
P(k<x_{\sfrac{1}{2}})=\frac{C}{1-\gamma}k^{1-\gamma}\bigg\rvert_{x_{\min}}^{x_{\sfrac{1}{2}}}=\frac{6248}{-1}k^{-1}\bigg\rvert_{6248}^{x_{\sfrac{1}{2}}}=\frac{-6248}{x_{\frac{1}{2}}}-\frac{-6248}{6248}=\frac{1}{2}\Rightarrow{\frac{6248}{x_{\frac{1}{2}}}}=\frac{1}{2}\Rightarrow\]
\[\Rightarrow{x_{\frac{1}{2}}=2\cdot6248}=12496
\]

Hence, the median wage is $12,496$ ILS, which can be considered close to the average wage $13,620$ ILS, which means that nearly half the population get the average wage or below.
\end{document}
