\documentclass{article}
\usepackage{graphicx} % Required for inserting images
\usepackage{float} % For the [H] float option
\usepackage{amsmath}
\usepackage{amssymb}
\usepackage{xfrac}
\title{Networks 2}
\author{Haim Lavi, 038712105}
\date{December 2025}

\begin{document}

\maketitle
\section{}
In $R(n,d)$, Each node $i$ has exactly $d$ neighbors. Starting from a random node $n_1$, we assume $n_1$ has $d$ nodes connected to it, namely $n_{1,1},n_{1,2},\dots,n_{1,d}$, but one of these nodes is $n_1$ itself, so we denote the new nodes by $n_{1,1},n_{1,2},\dots,n_{1,d-1}$. Each node $n_{1,i}$, $1\leq{i}\leq{d-1}$, has $d-1$ new nodes connected to it \[n_{1,i,1},n_{1,i,2},\dots,n_{1,i,d-1},\] 
and so on. 

Consider a random node $n_{i_1,i_2,\dots,i_k}$, where $k$ denotes the generation of the random node, and $1\leq{i_1,i_2,\dots,i_k}\leq{d-1}$, then the probability $u$ of $n_{i_1,i_2,\dots,i_k}$ not to ne in the $GCC$ is the probability of its $d-1$ new nodes, namely \[n_{i_1,i_2,\dots,i_k,1},n_{i_1,i_2,\dots,i_k,2},\dots,n_{i_1,i_2,\dots,i_k,d-1}\]
not to be in the $GCC$, that is $u=u^{d-1}$.
We have two trivial solutions, which are $u=0$ and $u=1$. 

When $u=1$ we can let the degree be any $d\geq{1}$, but $u=1$ means that the probability for any random node to be in the $GCC$ is $s=1-u=0$, hence no $GCC$ emerges for this solution. 

When $u=0$ we can let the degree be any $d\geq{2}$, because for $d=1$ we end up with $0=1$, which is obviously false. But for $d=2$ we get that any $u$ is a solution for this equation. But all this means that the $GCC$ emerges with high probability for any $d\geq{3}$, and with high probability it does not emerge for $d=1$, and for $d=2$ there is a probability $0<u<1$ for any random node not to be in the $GCC$, and therefore a probability $0<s<1$ for any random node to be in the $GCC$ (because $s=1-u$), so at $d=2$ we will have a phase transition, and we will have a large connected component which is not giant (meaning, the probability to be in this connected component does not tend to $1$).

The intuition is that for $d=1$, it is obvious that no GCC can be formed, because in that case we have only single edges in the network. But for any $d\geq{2}$, every node can be connected to the GCC.
\section{}
\subsection{}
For all $0\leq{a,b}\leq{N}$, the probability $\mathbb{P}[A_{ab}=1]=e^{-\alpha|b-a|}$, hence $p_{\max}=\mathbb{P}[A_{a,a+1}=1]=e^{-\alpha}$, and $p_{\min}=\mathbb{P}[A_{a,a+\frac{N}{2}}=1]=e^{-\alpha\frac{N}{2}}$ (the exponent is linear by the nodes distance).

A node $a$ has degree $k_a$ if out of $N-1$ potential neighbors it achieves exactly $k_a$ neighbors and does not achieve $N-1-k_a$ neighbors, hence this resembles the binomial distribution, only that $p$ for success depends on the distance between the nodes. Hence, the expected value of $k_a$ has a lower bound of $m=(N-1)p_{\min}=(N-1)e^{-\alpha\frac{N}{2}}$, and an upper bound of $M=(N-1)p_{\max}=(N-1)e^{-\alpha}$. 
\section{}
For this plot we construct a random ER network ($N=10^4$ and $p=10^{-3}$). We create the vector $(k_1,k_2,\dots,k_{N})$ by calculating $k_i=\sum_{j=1}^{N}A_{ij}$ for every $1\leq{i}\leq{N}$. Then we count the number $N_i$ of nodes whose degree is $k_i$ and calculate $P(k_i)=\frac{N_i}{N}$. We plot this graph in blue.

We calculate the Poisson distribution $P(k)=\frac{e^{-\lambda}\lambda^k}{k!}$, where $\lambda=pN$, for every $0\leq{k}\leq\max\{k_i\}$, and plot this graph in red.

We can see that the two graphs are very similar to each other.
\begin{figure}[H]
    \makebox[\linewidth]{
\includegraphics[width=1.85\linewidth]{degree_distribution.png}
    }
\end{figure}
\section{}
\subsection{}
\[
P(k)=Ck^{-\gamma}
\]

\[
E(k^n)=\int_1^{\infty}k^nCk^{-\gamma}dk=C\int_1^{\infty}k^{n-\gamma}dk=\frac{C}{n-\gamma+1}k^{n-\gamma+1}\bigg\rvert_1^{\infty}
\]

But
\[
C=\frac{\gamma-1}{k_{\min}^{1-\gamma}}
\]
In our case $k_{\min}=1$, hence $C=\frac{\gamma-1}{1^{1-\gamma}}=\gamma-1$, hence 

\[
E(k^n)=\frac{\gamma-1}{n-\gamma+1}k^{n-\gamma+1}\bigg\rvert_1^{\infty}
\]

If $n-\gamma\geq-1$ then $n-\gamma+1\geq{0}$ and then $k^{n-\gamma+1}\overset{k\rightarrow\infty}{\longrightarrow}\infty$, so $E(k^n)$ diverges for every $\gamma\leq{n+1}$. If $\gamma>n+1$ then $k^{n-\gamma+1}\overset{k\rightarrow\infty}{\longrightarrow{0}}$, so $E(k^n)$ is finite.
\subsection{}
The mean is $\mu=E(k)=E(k^1)$, so it diverges for $\gamma\leq{1+1}=2$ and converges for $\gamma>2$.

The variance is $\sigma^2=E(k^2)-E(k)^2$, so it diverges for $\gamma\leq{2+1}=3$ and converges for $\gamma>3$ (because then both $E(k)$ and $E(k^2)$ converge).
\section{}
\subsection{}
Minimum wage $x_{\min}=6,247.67\approx{6,248}$ ILS

Average wage $\overline{x}=13,620$ ILS

\[
\mu=E(x)=\frac{C}{2-\gamma}x^{2-\gamma}\bigg\rvert_{6248}^{\infty}\approx\overline{x}=13620
\]

But

\[
C=\frac{\gamma-1}{x_{\min}^{1-\gamma}}=\frac{\gamma-1}{6248^{1-\gamma}}
\]

Hence

\[
\frac{\gamma-1}{\gamma-2}\frac{6248^{2-\gamma}}{6248^{1-\gamma}}=\frac{\gamma-1}{\gamma-2}\frac{6248^{1-\gamma}6248}{6248^{1-\gamma}}=\frac{\gamma-1}{\gamma-2}6248\approx{13620}
\]

hence

\[
\frac{\gamma-1}{\gamma-2}\approx{2.18}
\]

hence

\[
\gamma-1\approx{2.18(\gamma-2)}=2.18\gamma-4.36
\]

hence

\[
1.18\gamma\approx{3.36}
\]

hence $\gamma\approx{2.847}$
\subsection{}
\[
x_{\max}\approx{N^{\frac{1}{\gamma-1}}}\approx(10^6)^{\frac{1}{2.847-1}}=10^{\frac{6}{1.847}}\approx{10^{3.248}}
\]

But this is an obvious mistake, so I will take the value of $\gamma$ to be around $2$, hence $x_{\max}\approx({10}^6)^{\frac{1}{2-1}}=10^6$.
\subsection{}
The wage distribution is close enough to continuous distribution, hence we need to find $x_{\sfrac{1}{2}}$ s.t. $P(x<x_{\sfrac{1}{2}})=\frac{1}{2}$, then $x_{\sfrac{1}{2}}$ is the median value.

\[
P(x<x_{\sfrac{1}{2}})=\int_{x_{\min}}^{\sfrac{1}{2}}Cx^{-\gamma}dx=\frac{C}{1-\gamma}x^{1-\gamma}\bigg\rvert_{x_{\min}}^{x_{\sfrac{1}{2}}}=\frac{1}{2}
\]

But in our case, $\gamma\approx2$, hence $C=\frac{\gamma-1}{x_{\min}^{1-\gamma}}\approx\frac{1}{6248^{-1}}=6248$, hence

\[
P(x<x_{\sfrac{1}{2}})=\frac{C}{1-\gamma}x^{1-\gamma}\bigg\rvert_{x_{\min}}^{x_{\sfrac{1}{2}}}\approx\frac{6248}{-1}x^{-1}\bigg\rvert_{6248}^{x_{\sfrac{1}{2}}}=\frac{-6248}{x_{\frac{1}{2}}}-\frac{-6248}{6248}=\frac{1}{2}\Rightarrow{\frac{6248}{x_{\frac{1}{2}}}}=\frac{1}{2}\Rightarrow\]
\[\Rightarrow{x_{\frac{1}{2}}\approx2\cdot6248}=12496
\]

Or, symmetrically

\[
P(x>x_{\sfrac{1}{2}})=\frac{C}{1-\gamma}x^{1-\gamma}\bigg\rvert_{x_{\frac{1}{2}}}^{\infty}\approx\frac{6248}{-1}x^{-1}\bigg\rvert_{x_{\frac{1}{2}}}^{\infty}=0-\frac{-6248}{x_{\frac{1}{2}}}=\frac{1}{2}\Rightarrow{{x_{\frac{1}{2}}\approx2\cdot6248}=12496}
\]

Hence, the median wage is $12,496$ ILS, which is lower than, but same order of magnitude as, the average wage $13,620$ ILS.
\subsection{}
We need to calculate how many people get more than $\overline{x}$

\[
P(x>\overline{x})=\frac{C}{1-\gamma}x^{1-\gamma}\bigg\rvert_{\overline{x}}^{\infty}=\frac{6248}{-1}x^{-1}\bigg\rvert_{\overline{x}}^{\infty}=0-\frac{-6248}{\overline{x}}=\frac{6248}{13620}\approx{0.458}
\]

But $0.458\cdot10=4.58$, hence there are $4$ deciles who get higher wages, hence the average wage is in the $5$th decile.
\subsection{}
We write $x\in{D_k}$ to say that $x$ is in the $k$th decile, and denote by $x_k$ the minimum wage in the $k$th decile (where $x_0=x_{\max}$ and $x_{10}=x_{\min}$). Then
\[
P(x\in{D_k})=P(x_k\leq{x}<x_{k-1})=\frac{C}{1-\gamma}x^{1-\gamma}\bigg\rvert_{x_{k}}^{x_{k-1}}=\frac{C}{1-\gamma}(x_{k-1}^{1-\gamma}-x_k^{1-\gamma})
\]
Whereas if $k=1$ then $P(x_k\leq{x}<x_{k-1})=P(x_1\leq{x}<x_0)=P(x_1\leq{x})$, because all people get less than $x_{\max}$ [hence $P(x<x_0)=1$], 

and if $k=10$ then $P(x_k\leq{x}<x_{k-1})=P(x_{10}\leq{x}<x_9)=P(x<x_9)$, because all people get more than $x_{\min}$ [hence $P(x\geq{x_{10}})=1$].

We use the value $\gamma\approx2$

\[
\frac{C}{1-\gamma}(x_{k-1}^{1-\gamma}-x_k^{1-\gamma})=-C(\frac{1}{x_{k-1}}-\frac{1}{x_k})=\frac{1}{10}\Rightarrow\frac{1}{x_k}-\frac{1}{x_{k-1}}=\frac{1}{10C}
\]
Hence

\[
\frac{1}{x_{k-1}}=\frac{1}{x_k}-\frac{1}{10C}\Rightarrow\frac{1}{x_{k-2}}=\frac{1}{x_{k-1}}-\frac{1}{10C}=(\frac{1}{x_k}-\frac{1}{10C})-\frac{1}{10C}=\frac{1}{x_k}-\frac{2}{10C}
\]

And in general

\[
\frac{1}{x_{k-i}}=\frac{1}{x_k}-\frac{i}{10C}\Rightarrow{x_{k-i}=\frac{10Cx_k}{10C-ix_k}}
\]

We use the value $C\approx6248=x_{\min}=x_{10}$, hence \[x_{10-i}=\frac{62480x_{10}}{62480-ix_{10}}=\frac{62480x_{\min}}{62480-ix_{\min}}=\frac{62480\cdot6248}{62480-i6248}=\]\[=\frac{10\cdot6248\cdot6248}{10\cdot6248-i6248}=\frac{6248(10\cdot6248)}{6248(10-i)}=\frac{62480}{10-i}
\]

For example $x_5=\frac{62480}{10-5}=\frac{62480}{5}=12496$, which is the median value that we calculated earlier.

We see that the sequence of differences \[\{a_k\}=\{x_9-x_{10},\dots,x_{k-1}-x_k,\dots,x_0-x_1\}\]
is monotone increasing, hence we know that $\max\{a_k\}=x_0-x_1=x_{\max}-x_1$, but we saw that $x_{\max}\approx10^6$, hence the highest difference is without proportion to the other differences, because we took $\gamma$ to be around $2$, hence $\sigma^2$ diverges.
\end{document}